\documentclass[letterpaper,aps,prc,superscriptaddress,nofootinbib,showpacs,floatfix]{revtex4-1}% PRC format\documentclass[12pt]
%\usepackage[super]{cite}
\usepackage{graphicx}
\usepackage{comment}
\usepackage{setspace}
%\usepackage{amsmath}
\usepackage{amssymb}
%\usepackage[textwidth=16cm,textheight=22cm]{geometry}
\usepackage{color}
\usepackage{indentfirst}
\usepackage{xspace}
\usepackage{hyperref}
\usepackage{verbatim}
\usepackage{epstopdf}

%\usepackage{lineno}
%\RequirePackage{lineno}
%\setlength{\linenumbersep}{6pt}
%\linenumber\cite{weib1,weib2} 
\newcommand{\sqs}{\mbox{$\sqrt{s}$}\xspace}
\newcommand{\ee}{\mbox{$e^{+} e^{-}$}\xspace}
\newcommand{\pp}{\mbox{$pp\,(p\bar{p})$ }\xspace}



\begin{document}

%\titile{
%in pp collisions at 13 TeV with Pythia 8}
\title{ Measurement of higher moments of   $\cdots$}
\author{XXX} 
\email{XXX@iitb.ac.in}
\affiliation{Indian Institute of Technology Bombay, Mumbai, India}
\author{YYY}
\affiliation{Indian Institute of Technology Bombay, Mumbai, India}
\author{ZZZ}
\affiliation{Indian Institute of Technology Bombay, Mumbai, India}

%\email{ashutosh.kumar.pandey@cern.ch}
%\email{sadhana@phy.iitb.ac.in}


\date{\today}  



\begin{abstract}

Write a proper brief abstarct ...
\end{abstract}

\maketitle

%%%%%%%%%%%%%%%%%%%%%%%%%%%%%%%%%%%%%%%%%%%%%%%%%%%%%%%%%%%%%%%%%%%%%
\section{Introduction}
%%%%%%%%%%%%%%%%%%%%%%%%%%%%%%%%%%%%%%%%%%%%%%%%%%%%%%%%%%%%%%%%%%%%%


The data provided is generated with Pythia 8  Monte Carlo event generator. \\
Number of events :  2 million \\
Collisions System :  p + p at centre of mass energy 13  TeV.\\


Define  the variables, moments and the used formula \\




\section{Experimental Observations }


1.  Distribution  of  $p_{T}$ and $<p_{T}>$ for different multiplicity classes (0-20, 20-40 etc).  \\

All the above plots should be in logarithmic scale. \\



2. For $|\eta < 2.5|$ (you may opt not to use any cut but mention in the text),  do the following  \\


a.  Plot $ \sqrt{ <\Delta p_{T1} \Delta p_{T2}>} / <p_{T}> $as a function of multiplicity class. \\

b.  Plot standardized skewness and intensive skewness as a function of multiplicity.\\

(Use the STAR definition. Please mention if you are using any other definition.)\\

Do not bother about errors. If possible plot the statistical errors .\\


\newpage
\begin{figure*}
\begin{center}
\includegraphics[scale=0.8]{ppspectra.eps}
\caption{(Color online) Put proper captions}
\label{f1}
\end{center}
\end{figure*}



\section{Summary}

The study of  $\cdots$


\begin{thebibliography}{50}
\medskip


%\bibitem{phenixwhite} K. ~Adcox 
  %Nuclear Phys. A{\bf 757},184-283 (2005). 
  
\bibitem{alicenature} J. ~Adams {\it et al.}, (ALICE Collaboration), Nature Physics{\bf 13},535-539 (2017). 


\end{thebibliography}

\end{document}

